\section{Подпространство}
\begin{definition}
    Нека $\langle X, \mathcal T_X \rangle$ е топологично пространство и $Y \subseteq X$.
    \begin{equation*}
        \mathcal T_Y = \left\{ U \cap Y \mid U \in \mathcal T_X \right\}
    \end{equation*}
    $\langle Y, \mathcal T_Y\rangle$ е топологично пространство и се нарича подпространство на $\langle X, \mathcal T_X \rangle$ породено от $Y$
\end{definition}
\begin{lemma}
    Подпространството е топология
\end{lemma}
\begin{proof}
    \cite[p.~89]{munkrestopology}
\end{proof}
\begin{lemma}
    Нека $\langle X, \mathcal T_X \rangle$ е топологично пространство и $\mathcal B_X$ е базис на $\mathcal T_X$.
    \begin{equation*}
        \mathcal B = \left\{ B \cap Y \mid B \in \mathcal B_X \right\}
    \end{equation*}
    $\mathcal B$ е базис за подпространството над $Y$
\end{lemma}
\begin{proof}
    \cite[p.~89]{munkrestopology}
\end{proof}
Интересно е кога едно множество е отворено в пространство и съответно в негово подпространство. Възможно е да имаме елемент на подпространството, т.ч. да не е елемент на голямото пространство. \lemref{lem:open-in-subspace} ни показва точно кога едно множество е отворено в двете пространства
\begin{lemma}\label{lem:open-in-subspace}
    Нека $Y$ е подпространство на $X$, $U$ е отворено в $Y$ и $Y$ е отворено в $X \Rightarrow U$ е отворено в $X$
\end{lemma}
\begin{proof}
    Лесно се вижда.

    $U$ е отворено в $Y \Rightarrow U \cap Y = V$ за някакво $V$ - отворено в $X$, но $Y$ е отворено в $X$. Значи $U$ е сечение на отворени мн-ва $\Rightarrow U$ е отворено.

    \cite[p.~89]{munkrestopology}
\end{proof}
\begin{lemma}
    Нека $Y$ е подпространство на $X$, $U$ е затворено в $Y \iff \exists V $ - затворено в $X:\; U = Y \cap V$
\end{lemma}
\begin{proof}
    \cite[p.~94]{munkrestopology}
\end{proof}
\begin{lemma}
    Нека $Y$ е подпространство на $X$, $U$ е затворено в $Y$ и $Y$ е затворено в $X \Rightarrow U$ е затворено в $X$
\end{lemma}
\begin{proof}
    \cite[p.~95]{munkrestopology}
\end{proof}

\begin{proposition}
    Нека $X$ е топологично пространство, $Y$ - произволно негово подпрстранство, a $Z \subseteq Y$ - произволно подмножество.

    Тогава топологията на подпространство на $Y$ породено от $Z$ и топологията на подпространство на $X$ породено от $Z$ съвпадат.
\end{proposition}

\subsection{Няколко примера}
\begin{example}
    Подпространството на $\R$ породено от $\N$. Това е точно дискретната топология над $\N$
\end{example}
\begin{proof}
    \begin{equation}
        \mathcal T = \left\{(a; b) \cap \N \mid a, b \in \R\right\}
    \end{equation}
    Това са всички естествени числа между произволни $a$ и $b$

    Кой да е синглетон $\{x\} \subseteq \N$ е елемент на $\mathcal T$, защото $(x-1;x+1) \cap \N = \{x\}$.

    Топологията $\mathcal T$ съдържа синглетони, значи е дискретната топология.
\end{proof}
\begin{example}
    Подпространството на $\R$ породено от $\Q$. Но това не е дискретна топология, понеже няма отворено множество в $\R$. 
\end{example}
\begin{example}
    Подпространството на $\R$ породено от $I = [0; 1]$ - единичния интервал.
\end{example}

\section{Връзка между операции}
\subsection{Подпространство и произведение}
\begin{theorem}
    Нека $X, Y$ - топологични пространства, а $A, B$ са съответно техни подпространства ($A \subseteq X, B \subseteq Y$). Произведението $A \times B$ като топологично пространство е същото, което $A \times B$ поражда като подпространство на $X \times Y$
\end{theorem}
\begin{proof}
    \cite[p.~87]{munkrestopology}
\end{proof}