\section{Общи съображения}
\subsection{Означения}
\begin{notation}
    Нека $\langle X, \mathcal T_X\rangle$ е топологично пространство. Когато е ясно от контекста, ще бележим пространството само с името на множеството - $X$
\end{notation}

\subsection{Познати обекти}
Известно е и топология породена от наредба:
\begin{definition}
    За тотално наредено пространство $X$ с релация $<$ и $a, b \in X: a \leq b$ следните наричаме \textbf{интервали}:
    \begin{equation*}
        \begin{split}
            (a, b) & = \left\{x\mid a < x < b\right\} \\
            [a, b) & = \left\{x\mid a \leq x < b\right\} \\
            (a, b] & = \left\{x\mid a < x \leq b\right\} \\
            [a, b] & = \left\{x\mid a \leq x \leq b\right\} \\
        \end{split}
    \end{equation*}
\end{definition}
\begin{notation}
    Използвайки означенията от горната дефиниция:
    \begin{itemize}
        \item $(a, b)$ наричаме \textbf{отворен интервал от $a$ до $b$}
        \item $[a, b]$ наричаме \textbf{затворен интервал от $a$ до $b$}
        \item $[a, b)$ наричаме \textbf{полуотворен интервал от $a$ до $b$} \textbf{затворен отляво}
        \item $(a, b]$ наричаме \textbf{полуотворен интервал от $a$ до $b$} \textbf{затворен отдясно}
    \end{itemize}
\end{notation}
Използването на думите не е случайно - ще се окаже, че отворените интервали са точно отворените множества на стандартната топология.
\begin{definition}
    Нека $X \neq \emptyset$ е мн-во с линейна наредба. Нека $m=\inf X, M=\sup X$ (ако съществуват). 

    \begin{equation*}
        \mathcal B_< = \left\{(a,b) \mid a < b\right\} \cup \underbrace{\left\{ [m, b) \mid b \in X \right\}}_{\text{ако $m$ съществува}} \cup \underbrace{\left\{ (a, M] \mid a \in X \right\}}_{\text{ако $M$ съществува}}
    \end{equation*}
\end{definition}
\begin{notation}
    $\mathcal B_<$ е базис на топология породена от наредба
\end{notation}

Когато се говори $\R$ и топология (без да е назована) се подразбира стандартната топология.
\begin{definition}
    Стандартна топология на $\R$ е
    \begin{equation*}
        \mathcal B = \{(a;b) \mid a<b\}
    \end{equation*}
    Топологията генерирана от всички отворени интервали.
\end{definition}

\subsection{Твърдения}
Посочените тук твърдения са от предишни теми и се използват наготово тук.

Интересно наблюдение е, че ако всеки синглетон принадлежи на топологията, то тя е еквивалентна на дискретната топология. Този факт се проверява като лесно следствие от дефиницията. $(\forall x \in X:\ \{x\} \in \mathcal T) \Rightarrow \mathcal T = \mathcal P(X)$. Обратното е тривиално.
\begin{lemma}
    Нека $\mathcal T$ е топология.

    \begin{equation*}
        (\forall x \in X:\ \{x\} \in \mathcal T) \iff \mathcal T = \mathcal P (X)
    \end{equation*}
\end{lemma}
\begin{proof}
    \begin{itemize}
        \item[$(\Rightarrow)$] Нека всеки синглетон е елемент на топологията.
        \begin{equation}
            \forall x \in X:\ \{x\} \in \mathcal T
        \end{equation}
        Значи всички крайни и безкрайни обединения на синглетони са елементи на топологията, което са всички крайни и безкрайни подмножества на $X$.
        
        Значи наистина $\mathcal T_X = \mathcal P (X)$

        \item[$(\Leftarrow)$] Нека $\mathcal T_X$ е дискретната топология
        \begin{equation}
            \mathcal T = \mathcal P (X)
        \end{equation}
        Тогава е тривиално наблюдението, че:
        \begin{equation}
            \forall x \in X:\; \{x\} \in \mathcal T
        \end{equation}
        Значи всеки синглетон принадлежи на топологията.
    \end{itemize}
\end{proof}