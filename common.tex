\section{Общи съображения}
\subsection{Означения}
\begin{notation}
    Нека $\langle X, \mathcal T_X\rangle$ е топологично пространство. Когато е ясно от контекста, ще бележим пространството само с името на множеството - $X$.
\end{notation}

\begin{definition}[Колмогорово пространство, $T_0$ пространство]
    $\langle X, \mathcal T_X \rangle$ е колмогорово пространство $\bydef\ \langle X, \mathcal T_X \rangle$ е топологично пространство и $\forall x,\ y \in X,\ \exists U \in \mathcal T_X:\; x \in U \oplus y \in U$

    Тук използваме $\oplus$ като изключваща дизюнкция (събиране по модул 2).

    Всеки две точки могат да се отделят една от друга чрез множество, което съдържа точно една от двете.
    \begin{figure}[H]
        \centering
        \begin{tikzpicture}[framed]
            \filldraw (-1, 0) circle (3pt) node [xshift = 2pt, anchor=west]{$x$};
            \filldraw (1, 0) circle (3pt) node [xshift = 2pt, anchor=west]{$y$};

            \draw (-1, 0) circle (1) node [xshift = 16pt, yshift = 16pt]{$U$};
        \end{tikzpicture}
        или 
        \begin{tikzpicture}[framed]
            \filldraw (-1, 0) circle (3pt) node [xshift = 2pt, anchor=west]{x};
            \filldraw (1, 0) circle (3pt) node [xshift = 2pt, anchor=west]{y};

            \draw (1, 0) circle (1) node [xshift = 16pt, yshift = 16pt]{$U$};
        \end{tikzpicture}
        \caption{Графично представяне}
    \end{figure}
\end{definition}
\begin{definition}
    Нека $\langle X, \mathcal T_X \rangle$ е $T_0$ пространство. $x, y \in X$ са топологично различими, ако удовлетворяват условието $\exists U \in \mathcal T_X:\; x \in U \oplus y \in U$
\end{definition}

\begin{definition}[Хаусдорфово пространство, $T_2$ пространство]
    $\langle X, \mathcal T_X \rangle$ е хаусдорфово пространство $\bydef\ \langle X, \mathcal T_X \rangle$ е топологично пространство и $\forall x_1,\ x_2 \in X,\ \exists U_1,\ U_2 \in \mathcal T_X:\; x_1 \in U_1,\ x_2 \in U_2, U_1 \cap U_2 = \emptyset$.

    Всеки две точки могат да се отделят една от друга чрез непресичащи се отворени множества.
    \begin{figure}[H]
        \centering
        \begin{tikzpicture}[framed]
            \filldraw (-1.5, 0) circle (3pt) node [xshift = 2pt, anchor=west]{x};
            \filldraw (1.5, 0) circle (3pt) node [xshift = 2pt, anchor=west]{y};

            \draw (-1.5, 0) circle (1) node [xshift = 16pt, yshift = 16pt]{$U$};
            \draw (1.5, 0) circle (1) node [xshift = 16pt, yshift = 16pt]{$V$};
        \end{tikzpicture}
        \caption{Графично представяне}
    \end{figure}
\end{definition}

\subsection{Познати обекти}
Известно ни е, че съществува и топология породена от наредба:
\begin{definition}
    За напълно наредено пространство $X$ с релация $<$ и $a, b \in X: a \leq b$ следните множества наричаме \textbf{интервали}:
    \begin{equation*}
        \begin{split}
            (a; b) & = \left\{x\mid a < x < b\right\} \\
            [a; b) & = \left\{x\mid a \leq x < b\right\} \\
            (a; b] & = \left\{x\mid a < x \leq b\right\} \\
            [a; b] & = \left\{x\mid a \leq x \leq b\right\} \\
        \end{split}
    \end{equation*}
\end{definition}
\begin{notation}
    Използвайки означенията от горната дефиниция:
    \begin{itemize}
        \item $(a; b)$ наричаме \textbf{отворен интервал от $a$ до $b$};
        \item $[a; b]$ наричаме \textbf{затворен интервал от $a$ до $b$};
        \item $[a; b)$ наричаме \textbf{полуотворен интервал от $a$ до $b$};\textbf{затворен отляво};
        \item $(a; b]$ наричаме \textbf{полуотворен интервал от $a$ до $b$} \textbf{затворен отдясно}.
    \end{itemize}
\end{notation}
Използването на думите не е случайно - ще се окаже, че отворените интервали са точно отворените множества на стандартната топология.
\begin{definition}
    Нека $X \neq \emptyset$ е мн-во с линейна наредба. Нека $m=\inf X, M=\sup X$ (ако съществуват). 

    \begin{equation*}
        \mathcal B_< = \left\{(a;b) \mid a < b\right\} \cup \underbrace{\left\{ [m; b) \mid b \in X \right\}}_{\text{ако $m$ съществува}} \cup \underbrace{\left\{ (a; M] \mid a \in X \right\}}_{\text{ако $M$ съществува}}
    \end{equation*}
\end{definition}
\begin{notation}
    $\mathcal B_<$ е база на топология породена от наредба.
\end{notation}

Когато се говори за $\R$ и топология (без да е назована), се подразбира стандартната топология.
\begin{definition}
    Стандартна топология на $\R$ е генерирана от $\mathcal B = \{(a;b) \mid a<b\}$.
    
    Т.е. топологията в $\R$ е генерирана от всички отворени интервали.
\end{definition}
Има много възможни топологии над $\R$. По-късно в текста ще бъде спомената и топологията на долната граница:
\begin{definition}
    Топология на долната граница на $\R$ се генерира от база $\mathbb S = \{[a; b)\mid a < b\}$.

    Също е известна като права на Зоргенфрей.
\end{definition}

\subsection{Функции}
\begin{definition}
    Нека $X, Y$ са топологични пространства.
    
    $p: X \to Y$ е непрекъсната $\bydef$ $\forall U \subseteq Y$ - отворено множество: $p^{-1}(U)$ е отворено в $X$.
\end{definition}
\begin{definition}
    Нека $X, Y$ са топологични пространства.
    
    $p: X \to Y$ е отворено изображение $\bydef$ $\forall U \subseteq X$ - отворено множество: $p(U)$ е отворено в $Y$.
\end{definition}
\begin{definition}
    Нека $X, Y$ са топологични пространства.
    
    $p: X \to Y$ е затворено изображение $\bydef$ $\forall U \subseteq X$ - затворено множество: $p(U)$ е затворено в $Y$.
\end{definition}
\begin{definition}
    Нека $X, Y$ са топологични пространства.
    
    $f: X \to Y$ е хомеоморфизъм $\bydef$ $f, f^{-1}$ - непрекъснати изображения.
\end{definition}
\begin{definition}
    Нека $X, Y$ са топологични пространства.
    
    $X$ е хомеоморфно на $Y$ $\bydef$ $\exists f: X \to Y$ - хомеоморфизъм.
\end{definition}
\begin{fact}
    Очевидно релацията "е хомеоморфно на"  е релация на еквивалентност:
    \begin{itemize}
        \item Всяко пространство $X$ е хомеоморфно на себе си със свидетел $id_X$;
        \item Ако $X$ е хомеоморфно на $Y$ със свидетел $f: X \to Y$. Знаем, че $f^{-1}: Y \to X$ също е хомеоморфизъм $\Rightarrow$ $Y$ е хомеоморфно на $X$;
        \item За транзитивност е ясно, че композицията на хомеоморфизми е хомеоморфизъм.
    \end{itemize}
\end{fact}

\subsection{Твърдения}
Посочените тук твърдения са от предишни теми и се използват наготово тук.

Интересно наблюдение е, че ако всяко едноточково множество принадлежи на топологията, то тя е еквивалентна на дискретната топология. Този факт се проверява като лесно следствие от дефиницията. $(\forall x \in X:\ \{x\} \in \mathcal T) \Rightarrow \mathcal T = \mathcal P(X)$. Обратното е тривиално.
\begin{lemma}
    Нека $\mathcal T$ е топология.

    \begin{equation*}
        (\forall x \in X:\ \{x\} \in \mathcal T) \iff \mathcal T = \mathcal P (X)
    \end{equation*}
\end{lemma}
\begin{proof}
    \begin{itemize}
        \item[$(\Rightarrow)$] Нека всяко едноточково множество е елемент на топологията:
        \begin{equation}
            \forall x \in X:\ \{x\} \in \mathcal T
        \end{equation}
        Значи всички крайни и безкрайни обединения на едноточкови множества са елементи на топологията, което са всички крайни и безкрайни подмножества на $X$.
        
        Значи наистина $\mathcal T_X = \mathcal P (X)$.

        \item[$(\Leftarrow)$] Нека $\mathcal T_X$ е дискретната топология:
        \begin{equation}
            \mathcal T = \mathcal P (X)
        \end{equation}
        Тогава е тривиално наблюдението, че:
        \begin{equation}
            \forall x \in X:\; \{x\} \in \mathcal T
        \end{equation}
        Значи всяко едноточково множество принадлежи на топологията.
    \end{itemize}
\end{proof}