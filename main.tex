\documentclass[11,a4paper]{article}
% \usepackage[margin=2cm,bottom=4cm]{geometry}
\addtolength{\oddsidemargin}{-.25in}
\addtolength{\evensidemargin}{-.25in}
\addtolength{\textwidth}{1.in}

\usepackage[T2A]{fontenc}
\usepackage[utf8]{inputenc}
\usepackage{setspace}

\usepackage[english, bulgarian]{babel}

\usepackage{fancyhdr}
\usepackage{footmisc}
\usepackage{calc}
\usepackage{titling}

\usepackage{float}
\usepackage{caption}

\usepackage{amsmath,amsthm,amssymb}
\usepackage{mathtext}
\usepackage{mathtools}

\usepackage{tikz}
\usepackage{stackengine}

\usepackage{enumerate}

\usepackage{array}
\usepackage[unicode]{hyperref}

\usepackage{biblatex}
\addbibresource{ref.bib}

\usepackage[draft]{pdfpages}

\hypersetup{
    colorlinks=true,
    linkcolor=blue,
    filecolor=magenta,      
    urlcolor=cyan,
}

% \setlength\headheight{3.5em}
% \setlength{\footskip}{2cm}

% \addtolength{\footnotesep}{0.8em} \renewcommand{\thefootnote}{\textbf{\arabic{footnote}}}

\renewcommand*{\thesection}{\arabic{section}}

\title{Топология - за извивките и ценителите}
\author{
  Цветелин Костадинов
}
\date{2024-02-22}

\pagestyle{fancy}
\fancyhf{}
\rhead{\thetitle}
\fancyfoot[R]{\thepage}
\lhead{
    \theauthor
}
\rfoot{\thepage}

\theoremstyle{plain}
\newtheorem{theorem}{Теорема}[section]
\newtheorem{problem}{Задача}[section]
\newtheorem{lemma}{{Лема}}[section]
\newtheorem{proposition}{Твърдение}[section]

\theoremstyle{definition}
\newtheorem{example}{Пример}[section]
\newtheorem{notation}{Означение}[section]
\newtheorem{definition}{Дефиниция}[section]
\newtheorem{corollary}{Следствие}[section]

\theoremstyle{remark}
\newtheorem{fact}{Факт}[section]
\newtheorem{remark}{{Забележка}}[section]
\newtheorem{solution}{Решение}[section]
\newtheorem{hint}{Упътване}[section]

% \def\proof{\textbf {Доказателство: }}%[section]
\renewcommand*{\proofname}{Доказателство}

\newcommand{\thref}[1]{\text{Теорема \ref{#1}}}
\newcommand{\lemref}[1]{\text{Лема \ref{#1}}}
\newcommand{\exampleref}[1]{\text{Пример \ref{#1}}}
\newcommand{\propref}[1]{\text{Твърдение \ref{#1}}}
\newcommand{\exerciseref}[1]{\text{Задача \ref{#1}}}
\newcommand{\solref}[1]{\text{Решение \ref{#1}}}
\newcommand{\remarkref}[1]{\text{Забележка \ref{#1}}}
\newcommand{\figref}[1]{\figurename~\ref{#1}}

\newcommand{\ubss}{{\overset{\text{УБСС}}{\Longrightarrow}}}
\newcommand{\N}{{\mathbb{N}}}
\newcommand{\Z}{{\mathbb{Z}}}
\newcommand{\Q}{{\mathbb{Q}}}
\newcommand{\R}{{\mathbb{R}}}
\newcommand{\bydef}{{\overset{def}{\iff}}}
\DeclareMathOperator*{\maxi}{max}
\DeclareMathOperator*{\dom}{dom}
\DeclareMathOperator*{\ran}{range}

\begin{document}
\maketitle
\tableofcontents
% \include{orgranizational}
% \include{set-theory}
% \include{introduction-topology}
% \include{operations-with-spaces}
% \include{last}

% \include{Topic3-Elizabeth}
\section{Общи съображения}
\subsection{Означения}
\begin{notation}
    Нека $\langle X, \mathcal T_X\rangle$ е топологично пространство. Когато е ясно от контекста, ще бележим пространството само с името на множеството - $X$.
\end{notation}

\subsection{Познати обекти}
Известно ни е, че съществува и топология породена от наредба:
\begin{definition}
    За тотално наредено пространство $X$ с релация $<$ и $a, b \in X: a \leq b$ следните наричаме \textbf{интервали}:
    \begin{equation*}
        \begin{split}
            (a; b) & = \left\{x\mid a < x < b\right\} \\
            [a; b) & = \left\{x\mid a \leq x < b\right\} \\
            (a; b] & = \left\{x\mid a < x \leq b\right\} \\
            [a; b] & = \left\{x\mid a \leq x \leq b\right\} \\
        \end{split}
    \end{equation*}
\end{definition}
\begin{notation}
    Използвайки означенията от горната дефиниция:
    \begin{itemize}
        \item $(a; b)$ наричаме \textbf{отворен интервал от $a$ до $b$};
        \item $[a; b]$ наричаме \textbf{затворен интервал от $a$ до $b$};
        \item $[a; b)$ наричаме \textbf{полуотворен интервал от $a$ до $b$};\textbf{затворен отляво};
        \item $(a; b]$ наричаме \textbf{полуотворен интервал от $a$ до $b$} \textbf{затворен отдясно}.
    \end{itemize}
\end{notation}
Използването на думите не е случайно - ще се окаже, че отворените интервали са точно отворените множества на стандартната топология.
\begin{definition}
    Нека $X \neq \emptyset$ е мн-во с линейна наредба. Нека $m=\inf X, M=\sup X$ (ако съществуват). 

    \begin{equation*}
        \mathcal B_< = \left\{(a;b) \mid a < b\right\} \cup \underbrace{\left\{ [m; b) \mid b \in X \right\}}_{\text{ако $m$ съществува}} \cup \underbrace{\left\{ (a; M] \mid a \in X \right\}}_{\text{ако $M$ съществува}}
    \end{equation*}
\end{definition}
\begin{notation}
    $\mathcal B_<$ е базис на топология породена от наредба.
\end{notation}

Когато се говори за $\R$ и топология (без да е назована), се подразбира стандартната топология.
\begin{definition}
    Стандартна топология на $\R$ е:
    \begin{equation*}
        \mathcal B = \{(a;b) \mid a<b\}
    \end{equation*}
    Топологията генерирана от всички отворени интервали.
\end{definition}
Има много възможни топологии над $\R$, също ще бъде спомената по-късно в текста и:
\begin{definition}
    Топология на долната граница на $\R$ с базис:
    \begin{equation*}
       \mathbb S = \{[a; b)\mid a < b\}
    \end{equation*}
    Също известна като права на Соргенфрей.
\end{definition}

\subsection{Функции}
\begin{definition}
    Нека $X, Y$ са топологични пространства.
    
    $p: X \to Y$ е непрекъсната $\bydef$ $\forall U \subseteq Y$ - отворено множество: $p^{-1}(U)$ е отворено в $X$.
\end{definition}
\begin{definition}
    Нека $X, Y$ са топологични пространства.
    
    $p: X \to Y$ е отворено изображение $\bydef$ $\forall U \subseteq X$ - отворено множество: $p(U)$ е отворено в $Y$.
\end{definition}
\begin{definition}
    Нека $X, Y$ са топологични пространства.
    
    $p: X \to Y$ е затворено изображение $\bydef$ $\forall U \subseteq X$ - затворено множество: $p(U)$ е затворено в $Y$.
\end{definition}
\begin{definition}
    Нека $X, Y$ са топологични пространства.
    
    $f: X \to Y$ е хомеоморфизъм $\bydef$ $f, f^{-1}$ - непрекъснати изображения.
\end{definition}
\begin{definition}
    Нека $X, Y$ са топологични пространства.
    
    $X$ е хомеоморфно на $Y$ $\bydef$ $\exists f: X \to Y$ - хомеоморфизъм.
\end{definition}
\begin{fact}
    Очевидно релацията "е хомеоморфно на"  е релация на еквивалентност:
    \begin{itemize}
        \item Всяко пространство $X$ е хомеоморфно на себе си със свидетел $id_X$;
        \item Ако $X$ е хомеоморфно на $Y$ със свидетел $f: X \to Y$. Знаем, че $f^{-1}: Y \to X$ също е хомеоморфизъм $\Rightarrow$ $Y$ е хомеоморфно на $X$;
        \item За транзитивност е ясно, че композицията на хомеоморфизми е хомеоморфизъм.
    \end{itemize}
\end{fact}

\subsection{Твърдения}
Посочените тук твърдения са от предишни теми и се използват наготово тук.

Интересно наблюдение е, че ако всеки синглетон принадлежи на топологията, то тя е еквивалентна на дискретната топология. Този факт се проверява като лесно следствие от дефиницията. $(\forall x \in X:\ \{x\} \in \mathcal T) \Rightarrow \mathcal T = \mathcal P(X)$. Обратното е тривиално.
\begin{lemma}
    Нека $\mathcal T$ е топология.

    \begin{equation*}
        (\forall x \in X:\ \{x\} \in \mathcal T) \iff \mathcal T = \mathcal P (X)
    \end{equation*}
\end{lemma}
\begin{proof}
    \begin{itemize}
        \item[$(\Rightarrow)$] Нека всеки синглетон е елемент на топологията:
        \begin{equation}
            \forall x \in X:\ \{x\} \in \mathcal T
        \end{equation}
        Значи всички крайни и безкрайни обединения на синглетони са елементи на топологията, което са всички крайни и безкрайни подмножества на $X$.
        
        Значи наистина $\mathcal T_X = \mathcal P (X)$.

        \item[$(\Leftarrow)$] Нека $\mathcal T_X$ е дискретната топология:
        \begin{equation}
            \mathcal T = \mathcal P (X)
        \end{equation}
        Тогава е тривиално наблюдението, че:
        \begin{equation}
            \forall x \in X:\; \{x\} \in \mathcal T
        \end{equation}
        Значи всеки синглетон принадлежи на топологията.
    \end{itemize}
\end{proof}
% Предишни
% 1 Множества - функции и релации. Декартово произведение на две множества.
% Крайни множества. Изброими и неизброими множества. Ринцип на рекурсивната дефиниция.
% 2 Безкрайни множества и аксиома за избора. Добре наредени множества. Принцип на максимума.
% 3 Топологични пространства. Методи за въвеждане на топологии.Непрекъснати функции и хомеоморфизми.

% ->>>> 4 Операции върху топологични пространства - подпространства, суми, произведения, фактор-пространства.

\section{Общи съображения}
\subsection{Означения}
\begin{notation}
    Нека $\langle X, \mathcal T_X\rangle$ е топологично пространство. Когато е ясно от контекста, ще бележим пространството само с името на множеството - $X$
\end{notation}

\subsection{Познати обекти}
Известно е и топология породена от наредба:
\begin{definition}
    За тотално наредено пространство $X$ с релация $<$ и $a, b \in X: a \leq b$ следните наричаме \textbf{интервали}:
    \begin{equation*}
        \begin{split}
            (a, b) & = \left\{x\mid a < x < b\right\} \\
            [a, b) & = \left\{x\mid a \leq x < b\right\} \\
            (a, b] & = \left\{x\mid a < x \leq b\right\} \\
            [a, b] & = \left\{x\mid a \leq x \leq b\right\} \\
        \end{split}
    \end{equation*}
\end{definition}
\begin{notation}
    Използвайки означенията от горната дефиниция:
    \begin{itemize}
        \item $(a, b)$ наричаме \textbf{отворен интервал от $a$ до $b$}
        \item $[a, b]$ наричаме \textbf{затворен интервал от $a$ до $b$}
        \item $[a, b)$ наричаме \textbf{полуотворен интервал от $a$ до $b$} \textbf{затворен отляво}
        \item $(a, b]$ наричаме \textbf{полуотворен интервал от $a$ до $b$} \textbf{затворен отдясно}
    \end{itemize}
\end{notation}
Използването на думите не е случайно - ще се окаже, че отворените интервали са точно отворените множества на стандартната топология.
\begin{definition}
    Нека $X \neq \emptyset$ е мн-во с линейна наредба. Нека $m=\inf X, M=\sup X$ (ако съществуват). 

    \begin{equation*}
        \mathcal B_< = \left\{(a,b) \mid a < b\right\} \cup \underbrace{\left\{ [m, b) \mid b \in X \right\}}_{\text{ако $m$ съществува}} \cup \underbrace{\left\{ (a, M] \mid a \in X \right\}}_{\text{ако $M$ съществува}}
    \end{equation*}
\end{definition}
\begin{notation}
    $\mathcal B_<$ е базис на топология породена от наредба
\end{notation}

Когато се говори $\R$ и топология (без да е назована) се подразбира стандартната топология.
\begin{definition}
    Стандартна топология на $\R$ е
    \begin{equation*}
        \mathcal B = \{(a;b) \mid a<b\}
    \end{equation*}
    Топологията генерирана от всички отворени интервали.
\end{definition}

\subsection{Твърдения}
Посочените тук твърдения са от предишни теми и се използват наготово тук.

Интересно наблюдение е, че ако всеки синглетон принадлежи на топологията, то тя е еквивалентна на дискретната топология. Този факт се проверява като лесно следствие от дефиницията. $(\forall x \in X:\ \{x\} \in \mathcal T) \Rightarrow \mathcal T = \mathcal P(X)$. Обратното е тривиално.
\begin{lemma}
    Нека $\mathcal T$ е топология.

    \begin{equation*}
        (\forall x \in X:\ \{x\} \in \mathcal T) \iff \mathcal T = \mathcal P (X)
    \end{equation*}
\end{lemma}
\begin{proof}
    \begin{itemize}
        \item[$(\Rightarrow)$] Нека всеки синглетон е елемент на топологията.
        \begin{equation}
            \forall x \in X:\ \{x\} \in \mathcal T
        \end{equation}
        Значи всички крайни и безкрайни обединения на синглетони са елементи на топологията, което са всички крайни и безкрайни подмножества на $X$.
        
        Значи наистина $\mathcal T_X = \mathcal P (X)$

        \item[$(\Leftarrow)$] Нека $\mathcal T_X$ е дискретната топология
        \begin{equation}
            \mathcal T = \mathcal P (X)
        \end{equation}
        Тогава е тривиално наблюдението, че:
        \begin{equation}
            \forall x \in X:\; \{x\} \in \mathcal T
        \end{equation}
        Значи всеки синглетон принадлежи на топологията.
    \end{itemize}
\end{proof}

\section{Подпространство}
\begin{definition}
    Нека $\langle X, \mathcal T_X \rangle$ е топологично пространство и $Y \subseteq X$.
    \begin{equation*}
        \mathcal T_Y = \left\{ U \cap Y \mid U \in \mathcal T_X \right\}
    \end{equation*}
    $\langle Y, \mathcal T_Y\rangle$ е топологично пространство и се нарича подпространство на $\langle X, \mathcal T_X \rangle$ породено от $Y$
\end{definition}
\begin{lemma}
    Подпространството е топология
\end{lemma}
\begin{proof}
    \cite[p.~89]{munkrestopology}
\end{proof}
\begin{lemma}
    Нека $\langle X, \mathcal T_X \rangle$ е топологично пространство и $\mathcal B_X$ е базис на $\mathcal T_X$.
    \begin{equation*}
        \mathcal B = \left\{ B \cap Y \mid B \in \mathcal B_X \right\}
    \end{equation*}
    $\mathcal B$ е базис за подпространството над $Y$
\end{lemma}
\begin{proof}
    \cite[p.~89]{munkrestopology}
\end{proof}
Интересно е кога едно множество е отворено в пространство и съответно в негово подпространство. Възможно е да имаме елемент на подпространството, т.ч. да не е елемент на голямото пространство. \lemref{lem:open-in-subspace} ни показва точно кога едно множество е отворено в двете пространства
\begin{lemma}\label{lem:open-in-subspace}
    Нека $Y$ е подпространство на $X$, $U$ е отворено в $Y$ и $Y$ е отворено в $X \Rightarrow U$ е отворено в $X$
\end{lemma}
\begin{proof}
    Лесно се вижда.

    $U$ е отворено в $Y \Rightarrow U \cap Y = V$ за някакво $V$ - отворено в $X$, но $Y$ е отворено в $X$. Значи $U$ е сечение на отворени мн-ва $\Rightarrow U$ е отворено.

    \cite[p.~89]{munkrestopology}
\end{proof}
\begin{lemma}
    Нека $Y$ е подпространство на $X$, $U$ е затворено в $Y \iff \exists V $ - затворено в $X:\; U = Y \cap V$
\end{lemma}
\begin{proof}
    \cite[p.~94]{munkrestopology}
\end{proof}
\begin{lemma}
    Нека $Y$ е подпространство на $X$, $U$ е затворено в $Y$ и $Y$ е затворено в $X \Rightarrow U$ е затворено в $X$
\end{lemma}
\begin{proof}
    \cite[p.~95]{munkrestopology}
\end{proof}

\begin{proposition}
    Нека $X$ е топологично пространство, $Y$ - произволно негово подпрстранство, a $Z \subseteq Y$ - произволно подмножество.

    Тогава топологията на подпространство на $Y$ породено от $Z$ и топологията на подпространство на $X$ породено от $Z$ съвпадат.
\end{proposition}

\subsection{Няколко примера}
\begin{example}
    Подпространството на $\R$ породено от $\N$. Това е точно дискретната топология над $\N$
\end{example}
\begin{proof}
    \begin{equation}
        \mathcal T = \left\{(a; b) \cap \N \mid a, b \in \R\right\}
    \end{equation}
    Това са всички естествени числа между произволни $a$ и $b$

    Кой да е синглетон $\{x\} \subseteq \N$ е елемент на $\mathcal T$, защото $(x-1;x+1) \cap \N = \{x\}$.

    Топологията $\mathcal T$ съдържа синглетони, значи е дискретната топология.
\end{proof}
\begin{example}
    Подпространството на $\R$ породено от $\Q$. Но това не е дискретна топология, понеже няма отворено множество в $\R$. 
\end{example}
\begin{example}
    Подпространството на $\R$ породено от $I = [0; 1]$ - единичния интервал.
\end{example}

\section{Връзка между операции}
\subsection{Подпространство и произведение}
\begin{theorem}
    Нека $X, Y$ - топологични пространства, а $A, B$ са съответно техни подпространства ($A \subseteq X, B \subseteq Y$). Произведението $A \times B$ като топологично пространство е същото, което $A \times B$ поражда като подпространство на $X \times Y$
\end{theorem}
\begin{proof}
    \cite[p.~87]{munkrestopology}
\end{proof}

\subsection{Топология породена от наредба и подпространство}
Нека имаме топология породена от наредба над $\langle X, < \rangle$ и едно подмножество $Y \subseteq X$. Множеството $Y$ е наредено от $< \mid_Y$. Значи има топология породена от наредба за $\langle Y, < \rangle$, но може да се случи така, че да не съвпада с топологията на $Y$ като подпространство на $X$.
\begin{example}
    Нека $X = \R, Y = [0; 1) \cup {2}$. Отбелязваме, че $Y \subseteq \R$.
    
    Топологията на $Y$ като подпространство $\mathcal T_{sub}$ съдържа $\{2\}$, защото може да се представи като сечение на отворено множество от стандартната топология на $\R$ с $Y$:
    \begin{equation}
        \{2\} = Y \cap \left(\frac{3}{2}; \frac{5}{2}\right)
    \end{equation}

    Обаче ако разгледаме топологията на $Y$ породена от стандартната наредба на $\R$ - $\mathcal T_<$ с базис $\mathcal B_<$-  ще открием, че $\{2\}$ не е отворено множество. Всеки елемент на $\mathcal B_<$, който съдържа 2 по необходимост ще съдържа и други числа. Нека $2 \in S, S \in \mathcal B_<$:
    \begin{equation}
        S = (a; 2] \Rightarrow 2 \neq \frac{a + 2}{2} \in S
    \end{equation}
    Значи няма как да е отворено множество в $\mathcal T_<$, защото всяко отворено множество ще съдържа числа по-малки от 2.
\end{example}

\subsubsection*{Има случаи, обаче, когато двете съвпадат:}
\begin{example}
    Нека $X = \R,\ Y = [0; 1] = I$. Известно ни е, че $I \subset \R$.

    Като подпространство $\langle I, \mathcal T_{sub}\rangle$ има базис:
    \begin{equation}
        \mathcal B_{sub} = \left\{(a; b) \cap I\mid a < b\right\}
    \end{equation}
    Знаем, че за $(a; b) \cap I$ имаме няколко случая:
    \begin{equation}
        (a; b) \cap I = \begin{cases}
            (a; b)      & a, b \in I \\
            [0; b)      & a \notin I, b \in I\; (\star)\\
            (a; 1]      & a \in I, b \notin I\; (\star)\\
            I           & a < 0, b > 1 \\
            \emptyset   & a, b < 0 \lor a, b > 1 \\
        \end{cases}
    \end{equation}

    По дефиниция всичките са отворени в $\langle I, \mathcal T_{sub} \rangle$, означените със $(\star)$ не са отворени в $\R$. 
    
    Но това формира базиса на топологията като подпространство, което съвпада с базиса на топологията породена от наредбата $<$ над $I$:
    \begin{equation}
         \mathcal B_< = \left\{(a,b) \mid 0 < a < b < 1\right\} \cup \left\{ [0, b) \mid b \in I \right\} \cup \left\{ (a, 1] \mid a \in I \right\}
    \end{equation}
\end{example}

\printbibliography

\end{document}
