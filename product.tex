\section{Произведение на топологии}
\begin{definition}
    Нека $\langle X, \mathcal T_X \rangle$ и $\langle Y,\mathcal T_Y \rangle$ са топологични пространства и нека:
    \begin{equation*}
        \mathcal B = \left\{U \times V \mid U \in \mathcal T_X, V \in \mathcal T_Y\right\}
    \end{equation*}
    Тогава $\mathcal B$ е базис на топологично пространство над $X \times Y$.
\end{definition}
Хубаво е да се провери, че наистина е базис.
\begin{proof}
    Проверява се лесно. При съмнение - \cite[p.~86]{munkrestopology}.
\end{proof}
Това е, ако имаме целите топологии, но в повечето случаи предпочитаме да работим с базиси над мн-вата, затова служи следната т-ма:
\begin{theorem}\label{thm:basis-of-times}
    Нека $\mathcal B$ и $\mathcal C$ са базиси на топологии над $X$ и $Y$ съответно. Тогава
    \begin{equation*}
        \mathcal D = \left\{ B \times C \mid B \in \mathcal B, C \in \mathcal C \right\}
    \end{equation*}
    Тогава $\mathcal D$ е базис на топология над $X \times Y$.
\end{theorem}
\begin{proof}
    \cite[p.~87]{munkrestopology}.
\end{proof}
\begin{example}
    Стандартната топология $\R^2$ е стандартната топология на $\R$ $\times$ стандартната топология на $\R$. Но това е голямо множество. \thref{thm:basis-of-times} ни казва, че можем да работим с базиса от всички отворени правоъгълници (а това е доста по-малко мн-во).
\end{example}
\begin{example}
    Още една известна топология е $\R_l \times \R_l$ с името "равнина на Соргенфрей".
\end{example}
\begin{notation} 
    Нека $A_i, i \in I$ са мн-ва за някакво индексно мн-во $I$.
    
    $\pi_k : \prod\limits_i A_i \to A_k$ е стандартната проекция на $k$-тия елемент от Декартовото произведение.
\end{notation}
Тривиално наблюдение е, че $\forall k \in I: \pi_k$ е сюрективна. Тогава можем да конструираме:
\begin{equation*}
    \pi_k^{-1}(U) = A_1 \times \dots \times A_{k-1} \times U \times A_{k+1} \times \dots \times A_k \text{ за } U \subseteq A_k
\end{equation*}
От тук произтича и следната теорема:
\begin{theorem}
    \begin{equation*}
        \mathcal B = \left\{ \pi_1^{-1}(U) \mid U \in \mathcal T_X \right\} \cup \left\{ \pi_2^{-1}(V) \mid V \in \mathcal T_Y \right\}
    \end{equation*}
    $\mathcal B$ е подбазис на топологията на $X \times Y$.
\end{theorem}
\begin{proof}
    Лесно се забелязва, че:
    \begin{equation*}
        \forall U \in \mathcal T_X, V\in \mathcal T_Y : \pi_1(U) \cap \pi_2(V) = U \times V
    \end{equation*}
    което е отворено в $X \times Y$.
    
    \cite[p.~88]{munkrestopology}.
\end{proof}

\subsection{Обобщение}
Дефинирахме операцията за 2 пространства, но нищо не ни спира да го обобщим за произволен брой:
\begin{definition}
    Нека $\mathcal X = \left\{X_i\right\}_{i \in I}$ е фамилия от топологични пространства.

    Дефинираме $\prod \mathcal X = \prod\limits_{i \in I}X_i$ да бъде топологията, която е произведение на пространствата от фамилията.
\end{definition}
\begin{proposition}
    Произведението на топологии е асоциативно.
\end{proposition}
\begin{proof}
    Тривиално се проверява използвайки асоциативността на Декартовото произведение.
\end{proof}
\begin{notation}
    Ако $\forall i \in I,\ X_i = X$ за някакво фиксирано $X$ и $|I| = n$, тогава $\prod \mathcal X $ бележим с $X^n$ и се нарича \textbf{$n$-та степен на $X$}. Ако $|I| = \aleph_0$, то $\prod \mathcal X $ бележим с $X^\omega$ (не успях да намеря превод на \emph{$\omega$-tuple}).
\end{notation}
Естествено наблюдение е, че всички доказани твърдения са верни за произволни произведения на пространства.

\subsubsection{Генериране на обобщеното произведение}
Интересно е как могат да се генерират произведенията. За това ни помага следното твърдение:
\begin{proposition}
    Нека $\mathcal X = \left\{X_i\right\}_{i\in I}$ е фамилия от пространства, а $\mathcal W = \left\{W_i\right\}_{i\in I}$, т.ч. $\forall i\in I:\; W_i$ е отворено в $X_i$ и само за краен брой $W \in \mathcal W,\; W \subseteq X \in \mathcal X$ е вярно, че $W \neq X$.

    Тогава $\mathcal W$ е базис за $\prod \mathcal X$.
\end{proposition}
\begin{proof}
    \cite[p.~77]{engelking1989general}.
\end{proof}