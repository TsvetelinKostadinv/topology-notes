\section{Сума на топологии}
За целите на сумата ни трябва фамилия $\mathcal X = \left\{X_i\right\}_{i \in I}$, където $I$ е крайно. Алтернативно $\mathcal X = \left\{X_i\right\}_{i=1}^{n}$. Но тази фамилия не може да е произволна - мн-вата от фамилията трябва да са две по две непресичащи се.
\begin{equation} \label{eq:disjoint-family}
    i,j \in I:\; X_i \cap X_j = \begin{cases}
        X_i       & , i=j     \\
        \emptyset & , i\neq j
    \end{cases}
\end{equation}
\begin{notation}
    Ще бележим с $\mathcal X$ фамилията:
    \begin{equation*}
        \mathcal X = \left\{X_i\right\}_{i \in I} = \left\{X_i\right\}_{i=1}^{n}
    \end{equation*}
    удовлетворяваща \eqref{eq:disjoint-family}.
\end{notation}

Ако имаме топологии $\left\{\mathcal T_i\right\}_{i \in I}$ тогава можем да дефинираме топология над $\mathcal X$.
\begin{definition}\label{def:sum-topologies}
    Сума на топологични пространства $\left\{\langle X_i, \mathcal T_i\rangle\right\}_{i \in I}$ ще наричаме топологичното пространство $\langle X, \mathcal T\rangle$, където $X = \bigcup\limits_{i\in I}X_i$ и $\mathcal T = \left\{ U \subseteq X \mid \forall i \in I:\; U \cap X_i \in \mathcal T_i \right\}$.
\end{definition}
\begin{fact}
    Отворените множества в сумата $\bigoplus \mathcal X$ са обединенията на отворени множества от отделните топологии.
\end{fact}
\begin{notation}
    Ще бележим сума на топологичните пространства $\mathcal X$ с $\bigoplus \mathcal X$, $\bigoplus\limits_{i\in I} X_i$ или $X_1 \oplus X_2 \oplus \dots \oplus X_n$ (когато топологията е ясна от контекста).
\end{notation}
Разбира се, хубаво е да се изясни, че името "сума на \textbf{топологии}" има смисъл и наистина удовлетворява дефиницията. Наистина "сума на топологии" е топология.

\begin{proposition}
    Ако $X$ може да бъде изразено като фамилия от взаимно непресичащи се отворени подмножества $\mathcal X$, то топологията над $X$ е точно $\bigoplus \mathcal X$.
\end{proposition}
\begin{proof}
    \cite[p.~75]{engelking1989general}.
\end{proof}
\begin{corollary}
    Значи ако имаме релация на еквивалентност $\sim$ над $X$, то $X/_\sim$ е фамилия от взаимно непресичащи се отворени подмножества, значи
    \begin{equation*}
        X = \bigoplus X/_\sim
    \end{equation*}
    Разбира се, това ще е особена релация на еквивалентност понеже класовете на еквивалентност ще трябва да са отворени множества.
\end{corollary}

\begin{proposition}
    Сумата на топологии е комутативна.
\end{proposition}
\begin{proof}
    Достатъчно е да се провери за сума на две пространства. С тривиална индукция се проверява за произволен брой.
\end{proof}

\begin{notation}
    Винаги имаме естественото включване познато от теория на категориите:
    \begin{equation*}
        i_k : X_i \hookrightarrow \bigoplus \mathcal X
    \end{equation*}
\end{notation}
Очевидно е, че $i_k$ е инекция (полезно уточнение)
\begin{proposition}
    $\forall i \in I:\; X_i$ е подпространство на $\bigoplus \mathcal X$
\end{proposition}
\begin{proof}
    Директна проверка на дефинициите.
\end{proof}

\subsection{Отслабване на ограничението}
В дефиницията на сума на топологии изискваме всички пространства да са взаимно непресичащи се, но това изглежда не е голямо ограничение понеже за всяка фамилия (не непременно взаимно непресичащи се множества) $\mathcal X$ можем да си построим фамилия от пространства $\mathcal X'$, хомеоморфни на дадените и да зададем $\bigoplus \mathcal X = \bigoplus \mathcal X'$. Например чрез изображението:
\begin{equation}
    \begin{split}
        X \in \mathcal X                    \\
        X \times \{s\} = X' \in \mathcal X' \\
        p_s : X' \to X                      \\
        p_s(x, s) = x                       \\
        p_s = \pi_1
    \end{split}
\end{equation}
Така всяка фамилия от множества има сума (с точност до хомеоморфизъм).