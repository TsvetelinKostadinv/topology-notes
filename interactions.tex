\section{Връзка между операции}
\subsection{Подпространство и произведение}
С въвеждането на нови операции е хубаво да се провери как се отнася с вече известни, затова помага следното твърдение.
\begin{proposition}
    Нека $X, Y$ - топологични пространства, а $A, B$ са съответно техни подпространства ($A \subseteq X, B \subseteq Y$). Произведението $A \times B$ като топологично пространство е същото, което $A \times B$ поражда като подпространство на $X \times Y$
\end{proposition}
\begin{proof}
    \cite[p.~87]{munkrestopology}
\end{proof}
И неговото обобщение:
\begin{proposition}
    Нека $\mathcal X = \left\{X_i\right\}_{i\in I}$ е фамилия от пространства, а $\mathcal A = \left\{A_i\right\}_{i\in I}$ е фамилия от подпространства, т.ч.
    \begin{equation}
        \forall i \in I:\; A_i \text{ е подпространство на } X_i
    \end{equation}
    Тогава топологията на произведението $\prod \mathcal A$ е същата като топологията на подпрстранството генерирано от $\prod \mathcal A$ като мн-во
\end{proposition}
\begin{proof}
    \cite[p.~78]{engelking1989general}
\end{proof}

\subsection{Подпространство и топология породена от наредба}
Нека имаме топология породена от наредба над $\langle X, < \rangle$ и едно подмножество $Y \subseteq X$. Множеството $Y$ е наредено от $< \mid_Y$. Значи има топология породена от наредба за $\langle Y, < \rangle$, но може да се случи така, че да не съвпада с топологията на $Y$ като подпространство на $X$.
\begin{example}
    Нека $X = \R, Y = [0; 1) \cup {2}$. Отбелязваме, че $Y \subseteq \R$.
    
    Топологията на $Y$ като подпространство $\mathcal T_{sub}$ съдържа $\{2\}$, защото може да се представи като сечение на отворено множество от стандартната топология на $\R$ с $Y$:
    \begin{equation}
        \{2\} = Y \cap \left(\frac{3}{2}; \frac{5}{2}\right)
    \end{equation}

    Обаче ако разгледаме топологията на $Y$ породена от стандартната наредба на $\R$ - $\mathcal T_<$ с базис $\mathcal B_<$-  ще открием, че $\{2\}$ не е отворено множество. Всеки елемент на $\mathcal B_<$, който съдържа 2 по необходимост ще съдържа и други числа. Нека $2 \in S, S \in \mathcal B_<$:
    \begin{equation}
        S = (a; 2] \Rightarrow 2 \neq \frac{a + 2}{2} \in S
    \end{equation}
    Значи няма как да е отворено множество в $\mathcal T_<$, защото всяко отворено множество ще съдържа числа по-малки от 2.
\end{example}

\subsubsection*{Има случаи, обаче, когато двете съвпадат:}
\begin{example}
    Нека $X = \R,\ Y = [0; 1] = I$. Известно ни е, че $I \subset \R$.

    Като подпространство $\langle I, \mathcal T_{sub}\rangle$ има базис:
    \begin{equation}
        \mathcal B_{sub} = \left\{(a; b) \cap I\mid a < b\right\}
    \end{equation}
    Знаем, че за $(a; b) \cap I$ имаме няколко случая:
    \begin{equation}
        (a; b) \cap I = \begin{cases}
            (a; b)      & a, b \in I \\
            [0; b)      & a \notin I, b \in I\; (\star)\\
            (a; 1]      & a \in I, b \notin I\; (\star)\\
            I           & a < 0, b > 1 \\
            \emptyset   & a, b < 0 \lor a, b > 1 \\
        \end{cases}
    \end{equation}

    По дефиниция всичките са отворени в $\langle I, \mathcal T_{sub} \rangle$, означените със $(\star)$ не са отворени в $\R$. 
    
    Но това формира базиса на топологията като подпространство, което съвпада с базиса на топологията породена от наредбата $<$ над $I$:
    \begin{equation}
         \mathcal B_< = \left\{(a,b) \mid 0 < a < b < 1\right\} \cup \left\{ [0, b) \mid b \in I \right\} \cup \left\{ (a, 1] \mid a \in I \right\}
    \end{equation}
\end{example}

\subsection{Сума на пространства и непрекъснати функции}
Редно е да проверим как сумата на пространства се отнася с непрекъснатите функции. Това ни дава точно 
\begin{theorem}
    За фамилия от непресичащи се пространства $\mathcal X$ и произволно пространство, ф-ията $f : \bigoplus \mathcal X \to Y$ е непрекъстната $\iff \forall s \in I:\; f \circ  i_s$ е непрекъсната
\end{theorem}
\begin{proof}
    \cite[p.~75]{engelking1989general}
\end{proof}

\subsection{Сума и затворени множества}
От дефиницията е ясно кои множества са отворени - обединенията на отворени множества от отделните топологии. От дефиницията на затворени множества е ясно, че можем да кажем кои са затворените, но това свойство се проверява трудно. Следната лема ни дава по-удобно за работа условие.
\begin{lemma}
    $U \subseteq \bigoplus \mathcal X$ е затворено $\iff \forall i \in I:\; U \cap X_i$ е затворено в $X_i$
\end{lemma}
\begin{proof}
    \cite[p.~74]{engelking1989general}
\end{proof}

\begin{corollary}
    $\forall i \in I:\; X_i$ е отворено и затворено в $\bigoplus \mathcal X$
\end{corollary}
\begin{proposition}
    Нека $\mathcal X$ е познатата фамилия от взаимно непресичащи се множества, а $\mathcal A = \{A_i\}_{i \in I}$ е фамилия от множества със свойството
    \begin{equation*}
        \forall i\in I:\; A_i \text{ е подпространство на } X_i
    \end{equation*}
    Нека $A = \bigcup \mathcal A$. Тогава топологията $\bigoplus \mathcal A$ съвпада с подпространството на $\bigoplus \mathcal X$ породено от $A$
\end{proposition}
\begin{proof}
    \cite[p.~75]{engelking1989general}
\end{proof}
